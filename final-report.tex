% v2-acmsmall-sample.tex, dated March 6 2012
% This is a sample file for ACM small trim journals
%
% Compilation using 'acmsmall.cls' - version 1.3 (March 2012), Aptara Inc.
% (c) 2010 Association for Computing Machinery (ACM)
%
% Questions/Suggestions/Feedback should be addressed to => "acmtexsupport@aptaracorp.com".
% Users can also go through the FAQs available on the journal's submission webpage.
%
% Steps to compile: latex, bibtex, latex latex
%
% For tracking purposes => this is v1.3 - March 2012

\documentclass[prodmode,acmtecs]{acmsmall} % Aptara syntax


% Package to generate and customize Algorithm as per ACM style
\usepackage[ruled]{algorithm2e}
\renewcommand{\algorithmcfname}{ALGORITHM}
\SetAlFnt{\small}
\SetAlCapFnt{\small}
\SetAlCapNameFnt{\small}
\SetAlCapHSkip{0pt}
\IncMargin{-\parindent}

% Metadata Information
% \acmVolume{9}
% \acmNumber{4}
% \acmArticle{39}
% \acmYear{2010}
% \acmMonth{3}

% Copyright
%\setcopyright{acmcopyright}
%\setcopyright{acmlicensed}
%\setcopyright{rightsretained}
%\setcopyright{usgov}
%\setcopyright{usgovmixed}
%\setcopyright{cagov}
%\setcopyright{cagovmixed}

% DOI
% \doi{0000001.0000001}

%ISSN
% \issn{1234-56789}

% Document starts
\begin{document}

% Page heads
\markboth{Y. Xie et al.}{Ranking Code Completion Candidates}

% Title portion
\title{Ranking Code Completion Candidates}
\author{Yanan Xie
\affil{yaxie@ucsc.edu}
Yifei Wu
\affil{ywu151@ucsc.edu}
Ziyi Chen
\affil{zchen139@ucsc.edu}
}

\begin{abstract}
Code completion is becoming a basic feature to all kinds of IDEs. It helps programmers to avoid typos and learn new languages fast. With the development of hardware performance and machine learning techniques, a lot of work could be done to further improve the code completion performance. Traditional code completion function lists all reserved words and variable names that match user’s input and then rank those matches in alphabetical order. We propose a new code completion candidate ranking method which aims at ranking those candidates with contextual syntax and semantic information. A code completion plugin is also built for Sublime Text - a very popular cross-platform code editor.
\end{abstract}


\maketitle


\section{Introduction}
\section{Related work}
\section{Ranking variables}
Ziyi
\section{Modeling syntax}
Yifei
\section{Hacking the editor}



This is to show how to attach figure and table.

\begin{figure}
\centerline{\includegraphics{acmsmall-mouse}}
\caption{Code before preprocessing.}
\label{fig:one}
\end{figure}

\begin{table}%
\tbl{Simulation Configuration\label{tab:one}}{%
\begin{tabular}{|l|l|}
\hline
TERRAIN{$^a$}   & (200m$\times$200m) Square\\\hline
Node Number     & 289\\\hline
Node Placement  & Uniform\\\hline
Application     & Many-to-Many/Gossip CBR Streams\\\hline
Payload Size    & 32 bytes\\\hline
Routing Layer   & GF\\\hline
MAC Layer       & CSMA/MMSN\\\hline
Radio Layer     & RADIO-ACCNOISE\\\hline
Radio Bandwidth & 250Kbps\\\hline
Radio Range     & 20m--45m\\\hline
\end{tabular}}
\begin{tabnote}%
\Note{Source:}{This is a table
sourcenote. This is a table sourcenote. This is a table
sourcenote.}
\vskip2pt
\Note{Note:}{This is a table footnote.}
\tabnoteentry{$^a$}{This is a table footnote. This is a
table footnote. This is a table footnote.}
\end{tabnote}%
\end{table}%

\section{Conclusions}

Just to show how to cite reference. \cite{Abril07},


% Acknowledgments
\begin{acks}
The authors would like to thank Dr. Maura Turolla of Telecom
Italia for providing specifications about the application scenario.
\end{acks}

% Bibliography
\bibliographystyle{ACM-Reference-Format-Journals}
\bibliography{acmsmall-sample-bibfile}
 


\end{document}
% End of v2-acmsmall-sample.tex (March 2012) - Gerry Murray, ACM


